\section{Related Work}
\label{ss:sec:related_work}

\subsection{Trap Door Rug Pulls}
A recent systemization of knowledge~\cite{SoKRugPull2025} develops an overarching rug pull taxonomy and compiles the dataset we use in this work.
Various works focus on detection of these rug pulls.
Similar to us, TokenAuditor~\cite{tokenauditor_qrs22} uses a fuzzing-based approach.
Pied Piper~\cite{pied-piper} first performs static analysis with datalog to detect trap door candidates and then validates them through directed fuzzing.
Several works~\cite{do-not-rug_math22,trade-or-trick_acs22,trapdoor} find rug pulls through heuristics and machine learning techniques.
Two of these works~\cite{do-not-rug_math22,trapdoor} center on tokens which were listed on Uniswap~\cite{uniswap_protocol}.
These works focus on particular features of rug pulls in order to unearth them.
However, our approach and tool look for general goal violations within ERC-20 contracts.
With the exception of the state initialization procedure, we do not tailor our search to rug pulls in particular.
This frees us from dependence on particular characteristics of rug pull smart contracts.

\subsection{Smart Contract Fuzzers and State Initialization}
Wu et al present a recent survey on smart contract fuzzers~\cite{fuzzer_survey_2024}.
Similar to us, several works use static and dynamic dataflow-based guidance~\cite{smartian_ase21,ethploit,ir-fuzz_ifs23} to construct targeted transaction sequences.
Additionally, many fuzzers use code coverage~\cite{sfuzz_icse20,ityfuzz_issta23,reguard} and branch distance~\cite{sfuzz_icse20,ir-fuzz_ifs23,ityfuzz_issta23} in order to prioritize interesting seeds. 
Combining fuzzing and symbolic execution, some works leverage symbolic analysis in order to solve difficult-to-reach constraints~\cite{ConFuzzius_eurosp21,ilf_ccs19}.
Other works employ machine learning techniques to guide transaction sequence generation~\cite{rlf_ase22,ilf_ccs19}.
In general, these fuzzers are oriented at discovering bugs and vulnerabilities, which is a distinct task from finding goal violations.
While these techniques are useful to improve our tool, our task required a custom fuzzer to tightly integrate the role/goal specification format, including optimization and call capabilitiy goals.
IR-Fuzz~\cite{ir-fuzz_ifs23} highlights that many fuzzers expend excessive energy around state initialization.
Similarly, our state initialization methodology is oriented at finding goal violations more efficiently.
However, our aim is to find usage-realistic initial states that are conducive to goal analysis, rather than conducive to finding bugs and vulnerabilities.

\subsection{Testing Beyond Traditional Bugs and Vulnerabilities in Smart Contracts}
There are two research directions which move beyond traditional bugs and vulnerabilities.
The first targets economic security, searching for profit-generating exploits~\cite{DeFiSoK2023,ClockworkFinance} rather than incorrect implementations.
The Clockwork Finance Framework~\cite{ClockworkFinance} employs formal techniques to compose user-defined models of the underlying mechanisms implemented through smart contracts in order to reason about their economic security properties.
Lanturn~\cite{Lanturn} uses adaptive learning directly on EVM bytecode to generate profitable transaction sequences.
The second research direction is in fairness verification.
Zeus~\cite{zeus_ndss18} searches for violations of specified business logic within implementations of smart contracts.
FairCon~\cite{fse20-fairness} uses properties from mechanism design literature to evaluate the fairness of auctions in DeFi.
FairChecker~\cite{fairchecker} proposes two general properties which detect if attackers may gain profit without cost.
Both directions have overlap with our task of finding end-user goal violations as what makes economic attacks and fairness violations concerning is that they violate end-user goals.
Our task differs from economic security as it is additionally concerned with non-economic goals, such as voting outcomes in DeFi governance.
It differs from fairness verification in that we are interested in identifying goal violations even if the underlying protocol is fair as end-users may not have accurate understandings of these protocols.
This is motivated by prior work~\cite{mentalmodel-soups,chi-end-user} which shows that end-users often misunderstand important mechanics of the blockchain and smart contracts.