Rug pulls are a prominent source of DeFi fraud, allowing malicious actors to profit at the expense of investors.
We investigate whether specifications of generic ERC-20 user roles (e.g., \holder) and goals (e.g., \maxowned) can be used to surface rug pull mechanisms in a high-level dataset of 1,376 known rug pull contracts.
These roles and goals are defined in terms of the contracts' state variables and transaction history, without tailoring to particularities of rug pulls.
To this end, we implement the \spoilsport fuzzer to automatically discover 2,731 transaction sequences which violated user goals.
Through manual analysis, we find that 83.6\% of rug pull types had corresponding violated goals.
The goal violations also revealed 105 misclassifications in the original dataset and unearthed 964 additional rug pull mechanisms that were not labelled in the dataset.
These goal-violating transaction sequences further serve as examples of how rug pulls may be performed for the given contracts.
Additionally, we develop a novel methodology for finding usage-realistic states upon which to begin our goal violation search, successfully initializing 1,324 (96.3\%) of our target contracts prior to beginning the search.