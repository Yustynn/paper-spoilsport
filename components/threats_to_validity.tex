\section{Threats to Validity}
\label{ss:sec:threats_to_validity}

\smallskip
\noindent \textbf{Limitations of the dataset:}
The original dataset is heavily skewed towards \textit{Transaction Limitation} (87\% of rug pulls)~\cite{SoKRugPull2025}.
After our filtering pipeline detailed in \autoref{ss:sec:evaluation_setup} and after excluding the 52 contracts which we failed to find initial state for, this skew increases to 98.3\%.
This limits the generalizability of our results to other rug pull types.
However, this is partially mitigated by our analysis in \textbf{RQ4}, which finds 964 non-\textit{Transaction Limitation} rug pull mechanics.
Additionally, this dataset was created from three composite datasets of other works~\cite{pied-piper,honeypot_usenix19,trapdoor}, implying that this may be representative of rug pull contracts.

\smallskip
\noindent
\textbf{Limitations of the goal violation discovery task:} 
    While this work demonstrates overlap, goal violations and rug pull mechanisms are not equivalent.
    As addressed in \autoref{ch:goalseeker}, there are legitimate purposes for goal violations which are not fraudulent.
    Further, end-users have subjective goals.
    While the specification is meant to be a reasonable stand in for this subjectivity, real-world end-user goals may differ from the specification.
    Additionally, ERC-20 contracts may have a wider diversity of roles.
    For instance, rug pull contracts often have rely on a router smart contract for integration with decentralized exchanges, which motivates a router role.
    Therefore the specification may be incomplete with respect to particular contracts.

\smallskip
\noindent
\textbf{Limitations of the \spoilsport solver:} 
We are primarily limited by our time budget and configuration of five candidates for each function parameter.
Additionally, we inherit the limitations of fuzzers in general, such as difficulty in finding test inputs which meet tightly-bounded conditions.
Specific to Ethereum, our fuzzer does not currently attempt different block timestamps for transactions.
This renders it difficult to find violations which are dependent on block timestamps.
Additionally, our solver does not support multi-contract analysis.
We therefore had to exclude 304 (12.7\% of) contracts which called external dependencies, introducing selection bias into our analysis.