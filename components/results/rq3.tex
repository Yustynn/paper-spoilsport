\smallskip
\noindent
\textbf{RQ3 (New Information on Rug Pull Mechanisms):}
We note that it is difficult to classify some features cleanly as rug pull mechanisms.
For instance, the \texttt{decreaseAllowance()} mechanic may be cleanly classified as a rug pull mechanic as its deceptive naming and implementation imply that it will be used fraudulently.
However, there are 12 contracts with transparently-named and correctly-implemented \texttt{pause()} functions which are all \textit{Anti-Motivated} (work against the transactor's own goals) as they globally halt all token transfers, including those by the attacker.
This could be used to halt token transfers while the attacker drains the accumulated paired tokens from a decentralized exchange, but it is also a feature that is commonly found in ERC-20 contracts~\cite{chi-end-user}.
While nearly all found goal violations may be used to further rug pulls, we do not provide a clean classification for this reason.

Goal violations provide four interesting sources of information during analysis.
The first is whether the goal is already violated upon state initialization.
1,253 \textit{sale-restrict} contracts are labelled under the \textit{sale-restrict} root cause, which is effected by disallowing all non-privileged users from selling their tokens.
However, 90.4\% of these contracts were not immediately violated upon state initialization.
This informs us that such contracts are capable of allowing users to trade tokens normally prior to the attacker restricting sales.

The second and third are the functions used in transactions and the roles of the transactors that call them.
We note that the \textit{sale-restrict} \liquidity violations come from 16 different sequences of function calls.
Additionally, analyzing multiple goal violations within the same smart contract can reveal further detail.
For these \textit{sale-restrict} contracts, 105 of them each have two \liquidity violations of the \owner calling the non-standard function \texttt{transfernewun()} and any user role calling the seemingly-innocuous function \texttt{transferFrom()}.
Upon manual analysis, we note that the first call to \texttt{transferFrom()} restricts that first recipient from further reception through the \texttt{transfer} function.
This is misclassified under the dataset's taxonomy, as it uses a \textit{Blacklist} root cause, rather than the \textit{Sale-restrict} root cause.
This \texttt{Blacklist} implementation is particularly elegant because it keeps the on-chain data looking clean as calls to functions named \texttt{transferFrom} are not suspicious, and because it operates as an ordinary \texttt{transferFrom} function after the first call.
In practice, the first \texttt{transferFrom()} call is generally to a liquidity pair on a decentralized exchange (DEX).
This thus allows the owner to add liquidity and then prevent subsequent selling of tokens by preventing victim \holder{}s from selling their tokens to that DEX without restricting them from purchasing.
This then allows the owner to withdraw the accumulated purchase liquidity.
The \texttt{transfernewun()} function implements the ability for the \owner to manually change the contract address.

The fourth is the motivation analysis.
The \textit{Motivated} sequences signal that the underlying transactors' goals are aligned with the goal violations, making them particularly important to inspect.
964/989 of the \textit{Motivated} transaction sequences are \minsupply violations which allow the \owner to arbitrarily generate tokens.
While these should be \textit{Token Generation} features, they are not labelled as such in the dataset.
These use various levels of obfuscation to mislead end-users.
Two of these are transparently named \texttt{mint()}, signalling token generation capability.
954 use the name of an internal, not publicly-accessible function (\texttt{\_mint()} in the frequently-used OpenZeppelin~\cite{OpenZeppelinContracts} ERC-20 implementation, but make it accessible.
This makes the \texttt{\_mint()} functions seem innocuously inaccessible at first glance.
Most misleadingly, six of these use function names which signal different behavior, such as \texttt{\_BURN()}, \texttt{blacklist()} and \texttt{work()}.


\begin{results}
    Goal violations enrich the dataset by revealing 105 misclassifications, reveal 964 unlabelled \textit{Token Generation}-type rug pull mechanics, and provide examples of rug mechanics being used.
\end{results}