
\smallskip
\noindent
\textbf{RQ4 (Comparison to \approach{}):}


\begin{table}[htbp]
\centering
\caption{
    Comparison of \approach and \spoilsport on goal violation detection.
    `\# Viol. Contracts' reports the number of contracts with at least one goal violation. `\# Viol.' reports the total number of violation sequences found.
    `Time (s)' reports the total time for solving each goal in seconds, excluding state initialization.
    }
\begin{tabular}{|>{\centering\arraybackslash}m{1.5cm}|>{\centering\arraybackslash}m{2.5cm}|c|c|c|c|c|c|}
\hline
& & \multicolumn{3}{c|}{\textbf{\approach}} & \multicolumn{3}{c|}{\textbf{\spoilsport}} \\
\cline{3-8}
\textbf{Role} & \textbf{Goal} & \makecell{\textbf{\# Viol.}\\\textbf{Contracts}} & \makecell{\textbf{\#}\\\textbf{Viol.}} & \makecell{\textbf{Time}\\\textbf{(s)}} & \makecell{\textbf{\# Viol.}\\\textbf{Contracts}} & \makecell{\textbf{\#}\\\textbf{Viol.}} & \makecell{\textbf{Time}\\\textbf{(s)}} \\
\hline
\multirow{3}{1.2cm}{\centering\holder}
    & \maxowned & 58 (44\%) & 65 & 7525.9 & \textbf{63 (48\%)} & \textbf{89} & \textbf{17.6} \\
\cline{2-8}
    & \minsupply & 38 (29\%) & 39 & 3346.4 & \textbf{39 (30\%)} & \textbf{40} & \textbf{23.4} \\
\cline{2-8}
    & \liquidity & 23 (17\%) & 31 & 13514.2 & \textbf{38 (29\%)} & \textbf{53} & \textbf{1140.0} \\
\hline
\owner & \maxsupply & 53 (40\%) & 58 & 3120.4 & \textbf{59 (45\%)} & \textbf{85} & \textbf{12.5} \\
\hline
\multicolumn{2}{|l|}{\textbf{TOTAL}} & 76 (58\%) & 193 & 27507.0 & \textbf{82 (62\%)} & \textbf{267} & \textbf{1193.0} \\
\hline
\end{tabular}
\label{ss:table:rq4}
\end{table}

Despite the bounded nature of fuzzing compared to symbolic execution, \spoilsport performs significantly better even with a lower time budget (\textit{see} \autoref{ss:table:rq4}).
We attribute this primarily to path explosion of the \approach{} solver.
Overall, it finds 74 more goal violations (+38.3\%) while finishing ~23x faster.
Among the goals, the \liquidity goal is notably the slowest.
This is due to the \texttt{transfer} function typically having more control flow dependencies, which leads to a larger candidate space for transactions to prepend while constructing a solution.
Additionally, call capability goals also add more overhead as they require more transaction executions in order to ensure reachability across different arguments.
In the worst case, checking if the \liquidity goal is violated takes as many transactions as users in our bounded address space.

\begin{figure*}[h]
	\centering
	\includegraphics[width=\linewidth]{img/gs_violations_by_goal_and_motivation.png}
	\caption{Counts of goal violations found by \spoilsport{} on the \approach dataset, broken down by goal and motivation label.}
	\label{ss:fig:rq4-violations-by-motivation}
\end{figure*}




We naturally find that the top code features responsible for each goal violation are the same as those found by \approach and have a similar motivation label distribution (\textit{see} \autoref{ss:fig:rq4-violations-by-motivation}).
The top responsible feature for both \maxsupply and \maxowned goal was \feature{burn}, accounting for 86\% and 82\% of their violations respectively.
For \minsupply{}, \feature{mint} accounted for 95\% of the goal violations.
For \liquidity{}, \feature{pause} accounted for 69\% of the goal violations.
Of note, \spoilsport discovered a novel though unconcerning \feature{mint}-based violations for \liquidity in the \textit{RigoToken} (\texttt{0x4fbb\allowbreak 3500\allowbreak 52bc\allowbreak a541\allowbreak 7566\allowbreak f188\allowbreak eb2e\allowbreak bce5\allowbreak b19b\allowbreak c964}) contract. 
By minting an extremely large amount of tokens to a target user, that user is no longer able to make the normally benign and inert transfer of their own full holdings to themselves.
This triggers an integer overflow and causes the transaction to revert.


\begin{results}
    \spoilsport finds more goal violations (+38.3\%) while finishing much faster (~23x) than \approach{}.
\end{results}