\smallskip
\noindent
\textbf{RQ1 (State Initialization):}



\begin{table}[h]
\centering
\caption{State initialization timing statistics (in seconds).}
\label{tab:rq1-timing}
\begin{tabular}{@{}lrrrrr@{}}
\toprule
& Mean & Median & Min & Max & Std \\
\midrule
Time (s) & 1.54 & 0.52 & 0.03 & 29.73 & 3.83 \\
\bottomrule
\end{tabular}
\end{table}

\spoilsport successfully initialized 1,324 of 1,376 (96.3\%) ERC-20 contracts with a timeout of 30 seconds.
\autoref{tab:rq1-timing} summarizes the timing statistics, showing a quick mean time (1.54s).

Of the 51 failures (3.7\%), 38 contracts (74.5\% of failures) were due to the solver timing out.
An further 13 contracts (25.5\% of failures) failed due to issues accessing or identifying the \texttt{balances} variable.

As shown in \autoref{ss:fig:rq1-init-length-distro} lengths ranged from two to twelve transactions, with the most frequent being lengths of three (940 contracts, 71.1\%), five (164 contracts, 11.9\%) and four (98 contracts, 7.1\%).
The vast majority (1,316, 98.9\%) of these sequences consisted solely of \texttt{transfer} calls.
This is consistent with the general pattern of all token being minted to a single address upon deployment, and that address then disbursing the funds.

\begin{figure*}[h]
	\centering
	\includegraphics[width=0.9\linewidth]{img/rp_state_init_sequence_length_distribution.png}
	\caption{Distribution of transaction sequence lengths for found state initialization sequences.}
	\label{ss:fig:rq1-init-length-distro}
\end{figure*}



\begin{figure}[h]
\centering
\resizebox{\linewidth}{!}{%
\begin{tikzpicture}[
    node distance=1cm,
    txstep/.style={rectangle, draw, rounded corners, align=center, font=\footnotesize, minimum width=1.6cm, minimum height=0.9cm},
    call/.style={->, >=stealth, thick}
]
    % Each node shows Caller.function\n(args)
    \node[txstep, fill=red!20] (s1) {\texttt{Addr2.DEPLOY}\\\texttt{(name, symbol, supply)}};
    \node[txstep, right=of s1] (s2) {\texttt{Addr1.transferFrom}\\\texttt{(Addr2, Addr1, amt)}};
    \node[txstep, right=of s2, fill=red!20] (s3) {\texttt{Addr2.\_mints}\\\texttt{(Addr4, amt)}};
    \node[txstep, right=of s3] (s4) {\texttt{Addr2.transfer}\\\texttt{(Addr0, amt)}};
    \node[txstep, right=of s4] (s5) {\texttt{Addr4.transfer}\\\texttt{(Addr3, amt)}};

    % Arrows
    \draw[call] (s1) -- (s2);
    \draw[call] (s2) -- (s3);
    \draw[call] (s3) -- (s4);
    \draw[call] (s4) -- (s5);
\end{tikzpicture}%
}
\caption{Example non-standard state initialization transaction sequence for the \textit{MrMr} token, with each node corresponding to a transaction.
The red nodes make use of rug pull mechanics.}
\label{ss:fig:rq1-mrmr-flow}
\end{figure}

In \autoref{ss:fig:rq1-mrmr-flow}, we highlight an example of a non-standard state initialization sequence found in the \textit{MrMr} contract (\texttt{0x01{\allowbreak}bcc8{\allowbreak}8d1b{\allowbreak}62e9{\allowbreak}7f9b{\allowbreak}d6a0{\allowbreak}58d1{\allowbreak}f442{\allowbreak}226a{\allowbreak}234ea3}) to show the generality of the state initialization search.
Here, the \owner{}'s (\textit{Address2}) deployment implicitly blocks all transfers that are not by themselves or a particular hardcoded address (\textit{Address1}) and mints themselves the initial token supply.
\textit{Address1} then calls a misleading \texttt{transferFrom} function, which succeededs because it passes the two checks of being whitelisted and transferring to the \owner.
The \owner then uses the misleadingly-named \texttt{\_mints()} function to whitelist \textit{Address4}, allowing it to transfer tokens, and overwrites its balances with a large amount rather than minting tokens.
This balance overwriting is covert and does not modify \texttt{totalSupply}.
\texttt{Address2} and \texttt{Address4} then transfer tokens to the remaining two users.
Our solver has now found a novel path which exploits rug pull mechanics to meet the target characteristics of token disbursal with at least one liquidity user.

This demonstrates that the declarative state specification using role/goal semantics is effective (96.3\% success) and efficient (median 0.52 seconds) for large-scale analysis, even in the adversarial setting of compromised ERC-20 contracts found in the wild.

\begin{result}
    \spoilsport successfully initialized 96.3\% (1,324) ERC-20 contracts quickly (mean 1.54s).
\end{result}