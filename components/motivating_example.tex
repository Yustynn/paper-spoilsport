\section{Motivating Example}
We use the example of the \textit{MrMr} contract
(\texttt{0x01{\allowbreak}bcc8{\allowbreak}8d1b{\allowbreak}62e9{\allowbreak}7f9b{\allowbreak}d6a0{\allowbreak}58d1{\allowbreak}f442{\allowbreak}226a{\allowbreak}234ea3})
to motivate our work. This is a rug pull contract which our dataset classifies as a \textit{sale-restrict} trap door, meaning that it only allows privileged users to sell their tokens.
Our task is to automatically search for violations of user goals (e.g., owning a maximal amount of tokens), and then manually check if these violations unearth features which may be used in rug pulls.
In \textit{MrMr}, we discover an additional \textit{balance manipulation} trap door, which allows a privileged user to manipulate a target user's balances directly.

We supply the role/goal specification and solidity source code to \spoilsport{}, which is then processed under three main stages shown in \autoref{fig:spoilsport-architecture} and detailed below.


\begin{lstlisting}[language=Solidity, caption={A \textit{balance manipulation} trapdoor in the \textit{MrMr} contract}, label={lst:mints}]
function _mints(address spender, uint256 addedValue) public returns (bool) {
    require(msg.sender==owner||msg.sender==address
    (1089755605351626874222503051495683696555102411980));
    if(addedValue > 0) {balanceOf[spender] = addedValue*(10**uint256(decimals));}
    canSale[spender]=true;
    return true;
}
\end{lstlisting}


\smallskip
\noindent
\textbf{Static Analysis:}
\revs{
    The aim of \textit{Static Analysis} is to derive information to use which can reduce the search space for the fuzzer.
    To this end, we extract function summaries
}
% Function summaries are extracted
in order to guide transaction sequence construction.
For the \texttt{\_mints()} function shown in \autoref{lst:mints}, this summary details that the function may modify the \texttt{balanceOf} and \texttt{canSale} state variables and that its path is conditioned on the value of the \texttt{owner} state variable.

This allows us to guide the fuzzer towards modifying target variables.
For instance, \texttt{\_mints()} would be a candidate to modify the \texttt{balanceOf} (\texttt{balances}) variable.
If the fuzzer is unable to call \texttt{\_mints}, we may search through these function summaries for functions which may modify the \texttt{owner} variable so that \texttt{\_mints} may be callable.


\begin{figure}[h]
\centering
\resizebox{\linewidth}{!}{%
\begin{tikzpicture}[
    node distance=1cm,
    txstep/.style={rectangle, draw, rounded corners, align=center, font=\footnotesize, minimum width=1.6cm, minimum height=0.9cm},
    call/.style={->, >=stealth, thick}
]
    % Each node shows Caller.function\n(args)
    \node[txstep, fill=red!20] (s1) {\texttt{Addr2.DEPLOY}\\\texttt{(name, symbol, supply)}};
    \node[txstep, right=of s1] (s2) {\texttt{Addr1.transferFrom}\\\texttt{(Addr2, Addr1, amt)}};
    \node[txstep, right=of s2, fill=red!20] (s3) {\texttt{Addr2.\_mints}\\\texttt{(Addr4, amt)}};
    \node[txstep, right=of s3] (s4) {\texttt{Addr2.transfer}\\\texttt{(Addr0, amt)}};
    \node[txstep, right=of s4] (s5) {\texttt{Addr4.transfer}\\\texttt{(Addr3, amt)}};

    % Arrows
    \draw[call] (s1) -- (s2);
    \draw[call] (s2) -- (s3);
    \draw[call] (s3) -- (s4);
    \draw[call] (s4) -- (s5);
\end{tikzpicture}%
}
\caption{\centering State initialization transaction sequence for the \textit{MrMr} token, with each node corresponding to a transaction.
The red nodes make use of rug pull mechanics.}
\label{ss:fig:mrmr-state-init}
\end{figure}



\smallskip
\noindent
\textbf{State Initialization:}
\revs{
    The aim of \textit{State Initialization} is to find a good starting state upon which to begin searching for goal violations.
    This starting state should resemble real-world usage in order to capture realistic goal violations.
    For this, we
}
use our fuzzer to find a state initialization transaction sequence which set up the \textit{MrMr} contract with the target \revs{usage-realistic} state characteristics of funds disbursed to users and at least one liquid user.
This is done through using the function summaries to find appropriate candidates that may meet our target characteristics, and then attempting to run those.

Our found state initialization sequence is shown in \autoref{ss:fig:mrmr-state-init}.
\revs{
    Interestingly, the \textit{MrMr} state initialization squence already uses suspicious contract features which play a role in the rug pull.
    As we will see, the \texttt{\_mints()} function reappears in a later goal violation.
    From the start,
}
the \owner{}'s (\textit{Address2}) deployment implicitly blocks all transfers that are not by themselves or a particular hardcoded address (\textit{Address1}) and mints themselves the initial token supply.
\textit{Address1} then calls a misleading \texttt{transferFrom()} function, which succeededs because it passes the two checks of the caller being whitelisted and transferring to the \owner.
The \owner then uses the \texttt{\_mints()} function to simultaneously whitelist \textit{Address4}, allowing it to transfer tokens, and overwrite \textit{Address4}'s balances with a large amount of tokens.
This balance overwriting is covert and does not modify \texttt{totalSupply}.
\texttt{Address2} and \texttt{Address4} then transfer tokens to the remaining two users.

Our \revs{fuzzer} has now found a novel path which exploits rug pull mechanics to meet our target state characteristics.


\smallskip
\noindent
\textbf{Goal Violation Search:}
\revs{
    The aim of the \textit{Goal Violation Search} is to find goal violating transaction sequences which may, as a side effect, disclose rug pull mechanics.
    We conduct our goal violation search on the contract, starting after running our discovered state initialization sequence.
}
Similar to the \textit{State Initialization} step, we guide the fuzzer in constructing transaction sequences which violate the specified goals.

We discover one violation of the \liquidity goal, which encodes the expectation that a user is able to transfer their full token holdings to any other user.
This corresponds to the labelled \textit{sale-restrict} category and is found immediately with no need for additional transactions.
Inspecting this violation reveals that the contract meets both the unlabelled \textit{whitelist/blacklist} category and the labelled \textit{sale-restrict} category.

We also discover six violations of the \maxowned goal, which encodes a user's desire to own a maximal amount of tokens.
One violation corresponding to the \texttt{\_mints()} function,  revealing that the contract should also fall under the \textit{balance manipulation} category.
The additional five violations corresponded to integer overflow bugs, which could also be used to reduce a user's token holdings.

% For the \textit{MrMr} contract, we discover eight