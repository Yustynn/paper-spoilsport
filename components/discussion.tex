\section{Discussion}
\label{ss:sec:discussion}


\smallskip
\noindent
\textbf{Avoiding dependence on particular features:}
By sticking purely to generic user roles and goals, we were able to avoid particularities of features during our rug pull analysis.
This is a fundamentally different approach to existing detectors, which index on known features~\cite{pied-piper,do-not-rug_math22}.
By avoiding fitting to these features, our approach may be robust to novel rug pull patterns which existing detectors may miss.
This is supported by our ability to detect rug pull concerns which were beyond the original dataset's labels, as shown in \textbf{RQ3}.
However, a trade off is that our found goal violations require manual inspection in order to understand.

% \todo
% \smallskip
% \noindent
% \textbf{Comorbidity of risky features:}
% \todo
% We note that many rug pull contracts have multiple features which would both violate user expectations as well as facilitate rug pulls.


\smallskip
\noindent
\textbf{Unclear classification of rug pulls based solely on features:}
While we examine many clearly hidden features that imply malice, we note that other more cleanly implemented features can only be classified as malicious or benign based on usage.
For instance, the USD Tether~\cite{TetherWhitepaper} contract, which houses the most popular ERC-20 smart contract, is centrally controlled by a single wallet with a \feature{mint} feature that allows it to generate an arbitrary amount of tokens.
This could be used to turn a rug pull-style quick profit by generating and then selling tokens, but this is unlikely due to other incentives and no abuse has ever been recorded.
Thus on an implementation-level, USD Tether has the necessary features to actuate a rug pull, yet by usage it is clearly not a rug pull.
This highlights the importance of surfacing risky features rather than cleanly classifying a smart contract as a rug pull or not after-the-fact.
