\section{Conclusion}
\label{ss:sec:conclusion}
We demonstrate that an analysis of general ERC-20 user roles and goals in 1,376 known rug pull contracts automatically recovers and enriches many rug pull mechanics.
These perspectives are implemented in the \approach framework, without tailoring to particular rug pull patterns.
Overall, we found 2,731 goal violations, unveiling unlabelled and mislabelled mechanics which may be used to rug pull end-users.
To conduct this analysis, we implement the \spoilsport goal-directed fuzzer which discovers more (+38.3\%) goal violations than the symbolic execution-based \approach implementation while also running ~23x faster.
Additionally, we develop a declarative methodology to find usage-realistic starting states from which to search for goal violations and define realistic state characteristics of ERC-20 rug pulls.
Using a dataset of known rug-pull contracts~\cite{SoKRugPull2025}, our evaluation shows that \spoilsport{} successfully initializes 1,324/1,376 (96.3\%) of our target contracts to these starting states.
Future work includes extending the \spoilsport fuzzer to handle multi-contract systems in order to better investigate financial fraud in more complex DeFi applications.
Together with prior chapters, this demonstrates the feasibility and real-world utility of analyzing end-user perspectives in smart contracts.